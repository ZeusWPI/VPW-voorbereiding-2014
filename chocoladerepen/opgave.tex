
\section*{Chocoladerepen}

Alice en Bob hebben zelf een spelletje uitgevonden, waarbij ze $n$
chocoladerepen achter elkaar plaatsen. Alice begint de chocoladerepen één na één
op te eten van links naar rechts, en Bob doet hetzelfde van rechts naar links.
Alice en Bob eten even snel, en van elke chocoladereep is op voorhand geweten
hoeveel tijd er nodig is om ze op te eten. Als Alice of Bob een chocoladereep
volledig heeft opgegeten, dan begint zij/hij onmiddellijk de volgende op te
eten. Bij hun spelletje is het niet toegelaten om twee chocoladerepen
terzelfdertijd op te eten, chocoladerepen maar half op te eten, of pauzes in te
lassen. Als beiden op hetzelfde moment dezelfde chocoladereep willen opeten, dan
is Bob de hoffelijkheid zelve en laat Alice voorgaan.

\subsection*{Input}

Op de eerste lijn van de input vind je steeds een geheel getal. Dit stelt het
aantal gevallen voor. Per geval is er 1 lijn van gehele getallen. Elk getal
staat voor de tijd om deze chocoladereep op te eten. De lijn \texttt{3 4 5 2}
stelt bijvoorbeeld 4 chocoladerepen voor, met van links naar rechts 3, 4, 5 en 2
tijdseenheiden om ze op te eten.

\subsubsection*{Voorbeeldinput}

\begin{verbatim}
5
1 2 3 4 5
5 3 4 2

5 3
1
\end{verbatim}

\subsection*{Output}

Per geval verwachten we het eental repen dat Alice op at, en het aantal repen
dat Bob op kreeg, gescheiden door een spatie.

\subsubsection*{Voorbeeldoutput}

\begin{verbatim}
3 2
2 2
0 0
1 1
1 0
\end{verbatim}
